% Generated by Sphinx.
\def\sphinxdocclass{report}
\documentclass[letterpaper,10pt,english]{sphinxmanual}
\usepackage[utf8]{inputenc}
\DeclareUnicodeCharacter{00A0}{\nobreakspace}
\usepackage[T1]{fontenc}
\usepackage{babel}
\usepackage{times}
\usepackage[Bjarne]{fncychap}
\usepackage{longtable}
\usepackage{sphinx}


\title{qmath Documentation}
\date{March 11, 2012}
\release{0.3.0}
\author{Marco Abrate}
\newcommand{\sphinxlogo}{}
\renewcommand{\releasename}{Release}
\makeindex

\makeatletter
\def\PYG@reset{\let\PYG@it=\relax \let\PYG@bf=\relax%
    \let\PYG@ul=\relax \let\PYG@tc=\relax%
    \let\PYG@bc=\relax \let\PYG@ff=\relax}
\def\PYG@tok#1{\csname PYG@tok@#1\endcsname}
\def\PYG@toks#1+{\ifx\relax#1\empty\else%
    \PYG@tok{#1}\expandafter\PYG@toks\fi}
\def\PYG@do#1{\PYG@bc{\PYG@tc{\PYG@ul{%
    \PYG@it{\PYG@bf{\PYG@ff{#1}}}}}}}
\def\PYG#1#2{\PYG@reset\PYG@toks#1+\relax+\PYG@do{#2}}

\def\PYG@tok@gd{\def\PYG@tc##1{\textcolor[rgb]{0.63,0.00,0.00}{##1}}}
\def\PYG@tok@gu{\let\PYG@bf=\textbf\def\PYG@tc##1{\textcolor[rgb]{0.50,0.00,0.50}{##1}}}
\def\PYG@tok@gt{\def\PYG@tc##1{\textcolor[rgb]{0.00,0.25,0.82}{##1}}}
\def\PYG@tok@gs{\let\PYG@bf=\textbf}
\def\PYG@tok@gr{\def\PYG@tc##1{\textcolor[rgb]{1.00,0.00,0.00}{##1}}}
\def\PYG@tok@cm{\let\PYG@it=\textit\def\PYG@tc##1{\textcolor[rgb]{0.25,0.50,0.56}{##1}}}
\def\PYG@tok@vg{\def\PYG@tc##1{\textcolor[rgb]{0.73,0.38,0.84}{##1}}}
\def\PYG@tok@m{\def\PYG@tc##1{\textcolor[rgb]{0.13,0.50,0.31}{##1}}}
\def\PYG@tok@mh{\def\PYG@tc##1{\textcolor[rgb]{0.13,0.50,0.31}{##1}}}
\def\PYG@tok@cs{\def\PYG@tc##1{\textcolor[rgb]{0.25,0.50,0.56}{##1}}\def\PYG@bc##1{\colorbox[rgb]{1.00,0.94,0.94}{##1}}}
\def\PYG@tok@ge{\let\PYG@it=\textit}
\def\PYG@tok@vc{\def\PYG@tc##1{\textcolor[rgb]{0.73,0.38,0.84}{##1}}}
\def\PYG@tok@il{\def\PYG@tc##1{\textcolor[rgb]{0.13,0.50,0.31}{##1}}}
\def\PYG@tok@go{\def\PYG@tc##1{\textcolor[rgb]{0.19,0.19,0.19}{##1}}}
\def\PYG@tok@cp{\def\PYG@tc##1{\textcolor[rgb]{0.00,0.44,0.13}{##1}}}
\def\PYG@tok@gi{\def\PYG@tc##1{\textcolor[rgb]{0.00,0.63,0.00}{##1}}}
\def\PYG@tok@gh{\let\PYG@bf=\textbf\def\PYG@tc##1{\textcolor[rgb]{0.00,0.00,0.50}{##1}}}
\def\PYG@tok@ni{\let\PYG@bf=\textbf\def\PYG@tc##1{\textcolor[rgb]{0.84,0.33,0.22}{##1}}}
\def\PYG@tok@nl{\let\PYG@bf=\textbf\def\PYG@tc##1{\textcolor[rgb]{0.00,0.13,0.44}{##1}}}
\def\PYG@tok@nn{\let\PYG@bf=\textbf\def\PYG@tc##1{\textcolor[rgb]{0.05,0.52,0.71}{##1}}}
\def\PYG@tok@no{\def\PYG@tc##1{\textcolor[rgb]{0.38,0.68,0.84}{##1}}}
\def\PYG@tok@na{\def\PYG@tc##1{\textcolor[rgb]{0.25,0.44,0.63}{##1}}}
\def\PYG@tok@nb{\def\PYG@tc##1{\textcolor[rgb]{0.00,0.44,0.13}{##1}}}
\def\PYG@tok@nc{\let\PYG@bf=\textbf\def\PYG@tc##1{\textcolor[rgb]{0.05,0.52,0.71}{##1}}}
\def\PYG@tok@nd{\let\PYG@bf=\textbf\def\PYG@tc##1{\textcolor[rgb]{0.33,0.33,0.33}{##1}}}
\def\PYG@tok@ne{\def\PYG@tc##1{\textcolor[rgb]{0.00,0.44,0.13}{##1}}}
\def\PYG@tok@nf{\def\PYG@tc##1{\textcolor[rgb]{0.02,0.16,0.49}{##1}}}
\def\PYG@tok@si{\let\PYG@it=\textit\def\PYG@tc##1{\textcolor[rgb]{0.44,0.63,0.82}{##1}}}
\def\PYG@tok@s2{\def\PYG@tc##1{\textcolor[rgb]{0.25,0.44,0.63}{##1}}}
\def\PYG@tok@vi{\def\PYG@tc##1{\textcolor[rgb]{0.73,0.38,0.84}{##1}}}
\def\PYG@tok@nt{\let\PYG@bf=\textbf\def\PYG@tc##1{\textcolor[rgb]{0.02,0.16,0.45}{##1}}}
\def\PYG@tok@nv{\def\PYG@tc##1{\textcolor[rgb]{0.73,0.38,0.84}{##1}}}
\def\PYG@tok@s1{\def\PYG@tc##1{\textcolor[rgb]{0.25,0.44,0.63}{##1}}}
\def\PYG@tok@gp{\let\PYG@bf=\textbf\def\PYG@tc##1{\textcolor[rgb]{0.78,0.36,0.04}{##1}}}
\def\PYG@tok@sh{\def\PYG@tc##1{\textcolor[rgb]{0.25,0.44,0.63}{##1}}}
\def\PYG@tok@ow{\let\PYG@bf=\textbf\def\PYG@tc##1{\textcolor[rgb]{0.00,0.44,0.13}{##1}}}
\def\PYG@tok@sx{\def\PYG@tc##1{\textcolor[rgb]{0.78,0.36,0.04}{##1}}}
\def\PYG@tok@bp{\def\PYG@tc##1{\textcolor[rgb]{0.00,0.44,0.13}{##1}}}
\def\PYG@tok@c1{\let\PYG@it=\textit\def\PYG@tc##1{\textcolor[rgb]{0.25,0.50,0.56}{##1}}}
\def\PYG@tok@kc{\let\PYG@bf=\textbf\def\PYG@tc##1{\textcolor[rgb]{0.00,0.44,0.13}{##1}}}
\def\PYG@tok@c{\let\PYG@it=\textit\def\PYG@tc##1{\textcolor[rgb]{0.25,0.50,0.56}{##1}}}
\def\PYG@tok@mf{\def\PYG@tc##1{\textcolor[rgb]{0.13,0.50,0.31}{##1}}}
\def\PYG@tok@err{\def\PYG@bc##1{\fcolorbox[rgb]{1.00,0.00,0.00}{1,1,1}{##1}}}
\def\PYG@tok@kd{\let\PYG@bf=\textbf\def\PYG@tc##1{\textcolor[rgb]{0.00,0.44,0.13}{##1}}}
\def\PYG@tok@ss{\def\PYG@tc##1{\textcolor[rgb]{0.32,0.47,0.09}{##1}}}
\def\PYG@tok@sr{\def\PYG@tc##1{\textcolor[rgb]{0.14,0.33,0.53}{##1}}}
\def\PYG@tok@mo{\def\PYG@tc##1{\textcolor[rgb]{0.13,0.50,0.31}{##1}}}
\def\PYG@tok@mi{\def\PYG@tc##1{\textcolor[rgb]{0.13,0.50,0.31}{##1}}}
\def\PYG@tok@kn{\let\PYG@bf=\textbf\def\PYG@tc##1{\textcolor[rgb]{0.00,0.44,0.13}{##1}}}
\def\PYG@tok@o{\def\PYG@tc##1{\textcolor[rgb]{0.40,0.40,0.40}{##1}}}
\def\PYG@tok@kr{\let\PYG@bf=\textbf\def\PYG@tc##1{\textcolor[rgb]{0.00,0.44,0.13}{##1}}}
\def\PYG@tok@s{\def\PYG@tc##1{\textcolor[rgb]{0.25,0.44,0.63}{##1}}}
\def\PYG@tok@kp{\def\PYG@tc##1{\textcolor[rgb]{0.00,0.44,0.13}{##1}}}
\def\PYG@tok@w{\def\PYG@tc##1{\textcolor[rgb]{0.73,0.73,0.73}{##1}}}
\def\PYG@tok@kt{\def\PYG@tc##1{\textcolor[rgb]{0.56,0.13,0.00}{##1}}}
\def\PYG@tok@sc{\def\PYG@tc##1{\textcolor[rgb]{0.25,0.44,0.63}{##1}}}
\def\PYG@tok@sb{\def\PYG@tc##1{\textcolor[rgb]{0.25,0.44,0.63}{##1}}}
\def\PYG@tok@k{\let\PYG@bf=\textbf\def\PYG@tc##1{\textcolor[rgb]{0.00,0.44,0.13}{##1}}}
\def\PYG@tok@se{\let\PYG@bf=\textbf\def\PYG@tc##1{\textcolor[rgb]{0.25,0.44,0.63}{##1}}}
\def\PYG@tok@sd{\let\PYG@it=\textit\def\PYG@tc##1{\textcolor[rgb]{0.25,0.44,0.63}{##1}}}

\def\PYGZbs{\char`\\}
\def\PYGZus{\char`\_}
\def\PYGZob{\char`\{}
\def\PYGZcb{\char`\}}
\def\PYGZca{\char`\^}
\def\PYGZsh{\char`\#}
\def\PYGZpc{\char`\%}
\def\PYGZdl{\char`\$}
\def\PYGZti{\char`\~}
% for compatibility with earlier versions
\def\PYGZat{@}
\def\PYGZlb{[}
\def\PYGZrb{]}
\makeatother

\begin{document}

\maketitle
\tableofcontents
\phantomsection\label{index::doc}


qmath is a Python library for algebraic quaternion calculations. Quaternions operations and manipulations are implemented, including square and cubic roots evaluations.
Hurwitz quaternions can also be used. For Hurwitz quaternions modular operations are implemented.
Moreover, other algebraic tools are available, such as Cross Ratio and Moebius transformations.


\chapter{Installation}
\label{intro:installation}\label{intro::doc}\label{intro:qmath-s-documentation}

\section{Requirements}
\label{intro:requirements}
The following software is required before installing qmath:
\begin{itemize}
\item {} 
Python 2.x+ (\href{http://www.python.org/}{http://www.python.org/})

\item {} 
NumPy 1.x+ (\href{http://new.scipy.org/download.html}{http://new.scipy.org/download.html})

\end{itemize}


\section{Download and Installation}
\label{intro:download-and-installation}
The qmath setup files can be downloaded from the qmath download page or the Python Package Index. Download the source tarball from \href{http://pypi.python.org/pypi/qmath}{http://pypi.python.org/pypi/qmath}.
Open a shell. Unpack the tarball in a temporary directory (not directly in Python's site-packages). Commands:

\code{\$ tar zxf qmath-X.Y.Z.tar.gz}

X, Y and Z are the major and minor version numbers of the tarball.
Go to the directory created by expanding the tarball:

\code{\$ cd qmath-X.Y.Z}

Get root privileges:

\code{\$ su}

\code{(enter root password)}

To install for python type:

\code{\$ python setup.py install}

If the python executable isn't on your path, you'll have to specify the complete path, such as /usr/local/bin/python.


\chapter{Classes}
\label{classes:classes}\label{classes::doc}

\section{quaternion()}
\label{classes:quaternion}\index{quaternion (built-in class)}

\begin{fulllineitems}
\pysiglinewithargsret{\strong{class }\bfcode{quaternion}}{\emph{attitude}\optional{, \emph{matrix = np.identity(3)}}}{}
\end{fulllineitems}


The Quaternion class.

\emph{attitude} can be:
\begin{itemize}
\item {} 
a number (of any type, complex are included);

\end{itemize}

\begin{Verbatim}[commandchars=\\\{\}]
\PYG{g+gp}{\textgreater{}\textgreater{}\textgreater{} }\PYG{k+kn}{import} \PYG{n+nn}{qmath}
\PYG{g+gp}{\textgreater{}\textgreater{}\textgreater{} }\PYG{n}{qmath}\PYG{o}{.}\PYG{n}{quaternion}\PYG{p}{(}\PYG{l+m+mi}{1}\PYG{p}{)}
\PYG{g+go}{(1.0)}
\PYG{g+gp}{\textgreater{}\textgreater{}\textgreater{} }\PYG{n}{qmath}\PYG{o}{.}\PYG{n}{quaternion}\PYG{p}{(}\PYG{l+m+mi}{1}\PYG{o}{+}\PYG{l+m+mi}{1}\PYG{n}{j}\PYG{p}{)}
\PYG{g+go}{(1.0+1.0i)}
\end{Verbatim}
\begin{itemize}
\item {} 
a list or a numpy array of the components with respect to \emph{1, i, j} and \emph{k};

\end{itemize}

\begin{Verbatim}[commandchars=\\\{\}]
\PYG{g+gp}{\textgreater{}\textgreater{}\textgreater{} }\PYG{n}{qmath}\PYG{o}{.}\PYG{n}{quaternion}\PYG{p}{(}\PYG{p}{[}\PYG{l+m+mi}{1}\PYG{p}{,}\PYG{l+m+mi}{2}\PYG{p}{,}\PYG{l+m+mi}{3}\PYG{p}{,}\PYG{l+m+mi}{4}\PYG{p}{]}\PYG{p}{)}
\PYG{g+go}{(1.0+2.0i+3.0j+4.0k)}
\PYG{g+gp}{\textgreater{}\textgreater{}\textgreater{} }\PYG{n}{qmath}\PYG{o}{.}\PYG{n}{quaternion}\PYG{p}{(}\PYG{n}{np}\PYG{o}{.}\PYG{n}{array}\PYG{p}{(}\PYG{p}{[}\PYG{l+m+mi}{1}\PYG{p}{,}\PYG{l+m+mi}{2}\PYG{p}{,}\PYG{l+m+mi}{3}\PYG{p}{,}\PYG{l+m+mi}{4}\PYG{p}{]}\PYG{p}{)}\PYG{p}{)}
\PYG{g+go}{(1.0+2.0i+3.0j+4.0k)}
\end{Verbatim}
\begin{itemize}
\item {} 
a string of the form \emph{`a+bi+cj+dk'};

\end{itemize}

\begin{Verbatim}[commandchars=\\\{\}]
\PYG{g+gp}{\textgreater{}\textgreater{}\textgreater{} }\PYG{n}{qmath}\PYG{o}{.}\PYG{n}{quaternion}\PYG{p}{(}\PYG{l+s}{'}\PYG{l+s}{1+1i+3j-2k}\PYG{l+s}{'}\PYG{p}{)}
\PYG{g+go}{(1.0+1.0i+3.0j-2.0k)}
\end{Verbatim}
\begin{itemize}
\item {} 
a rotation about an axis using pairs (rotation angle, axis of rotation);

\end{itemize}

\begin{Verbatim}[commandchars=\\\{\}]
\PYG{g+gp}{\textgreater{}\textgreater{}\textgreater{} }\PYG{n}{qmath}\PYG{o}{.}\PYG{n}{quaternion}\PYG{p}{(}\PYG{l+m+mf}{0.5} \PYG{o}{*} \PYG{n}{math}\PYG{o}{.}\PYG{n}{pi}\PYG{p}{,} \PYG{p}{[}\PYG{l+m+mi}{0}\PYG{p}{,}\PYG{l+m+mi}{0}\PYG{p}{,}\PYG{l+m+mi}{1}\PYG{p}{]}\PYG{p}{)}
\PYG{g+go}{(0.968912421711+0.247403959255k)}
\end{Verbatim}
\begin{itemize}
\item {} 
a list whose components are Euler angles;

\end{itemize}

\begin{Verbatim}[commandchars=\\\{\}]
\PYG{g+gp}{\textgreater{}\textgreater{}\textgreater{} }\PYG{k+kn}{import} \PYG{n+nn}{math}
\PYG{g+gp}{\textgreater{}\textgreater{}\textgreater{} }\PYG{n}{qmath}\PYG{o}{.}\PYG{n}{quaternion}\PYG{p}{(}\PYG{p}{[}\PYG{l+m+mf}{0.0}\PYG{p}{,}\PYG{n}{math}\PYG{o}{.}\PYG{n}{pi} \PYG{o}{/} \PYG{l+m+mi}{6}\PYG{p}{,}\PYG{n}{math}\PYG{o}{.}\PYG{n}{pi} \PYG{o}{/} \PYG{l+m+mi}{3}\PYG{p}{]}\PYG{p}{)}
\PYG{g+go}{(0.836516303738+0.482962913145i+0.224143868042j-0.129409522551k)}
\end{Verbatim}
\begin{itemize}
\item {} 
a 3X3 rotation matrix. The matrix must be given as a numpy array.

\end{itemize}

\begin{Verbatim}[commandchars=\\\{\}]
\PYG{g+gp}{\textgreater{}\textgreater{}\textgreater{} }\PYG{k+kn}{import} \PYG{n+nn}{numpy} \PYG{k+kn}{as} \PYG{n+nn}{np}
\PYG{g+gp}{\textgreater{}\textgreater{}\textgreater{} }\PYG{n}{qmath}\PYG{o}{.}\PYG{n}{quaternion}\PYG{p}{(}\PYG{n}{np}\PYG{o}{.}\PYG{n}{array}\PYG{p}{(}\PYG{p}{[}\PYG{p}{[}\PYG{l+m+mi}{0}  \PYG{p}{,}\PYG{o}{-}\PYG{l+m+mf}{0.8} \PYG{p}{,}\PYG{o}{-}\PYG{l+m+mf}{0.6}\PYG{p}{]}\PYG{p}{,}
\PYG{g+gp}{... }                           \PYG{p}{[}\PYG{l+m+mf}{0.8}\PYG{p}{,}\PYG{o}{-}\PYG{l+m+mf}{0.36}\PYG{p}{,} \PYG{l+m+mf}{0.48}\PYG{p}{]}\PYG{p}{,}
\PYG{g+gp}{... }                           \PYG{p}{[}\PYG{l+m+mf}{0.6}\PYG{p}{,}\PYG{l+m+mf}{0.48} \PYG{p}{,}\PYG{o}{-}\PYG{l+m+mf}{0.64}\PYG{p}{]}\PYG{p}{]}\PYG{p}{)}\PYG{p}{)}
\PYG{g+go}{(0.707106781187i+0.565685424949j+0.424264068712k)}
\end{Verbatim}

\emph{matrix} is a 3X3 symmetric matrix defining the product in the quaternion algebra. The identity matrix is by default: this choiche yields to classical Hamilton quaternions. Different choiches of this matrix lead to generalized quaternion algebras. For example, if
\begin{quote}

\emph{matrix = np.array({[}{[}-1,0,0{]},{[}0,1,0{]},{[}0,0,-1{]}{]})}
\end{quote}

one gets the algebra of square matrices of order two.


\section{hurwitz()}
\label{classes:hurwitz}\index{hurwitz (built-in class)}

\begin{fulllineitems}
\pysiglinewithargsret{\strong{class }\bfcode{hurwitz}}{\emph{attitude}\optional{, \emph{matrix = np.identity(3)}}}{}
\end{fulllineitems}


The class of Hurwitz quaternions, i.e. quaternions whose components are integers.

\emph{attitude} can be the same as for quaternion class, if the components as a quaternion are integers.


\chapter{Methods}
\label{methods::doc}\label{methods:methods}

\section{Basic Methods}
\label{methods:basic-methods}

\subsection{\_\_repr\_\_()}
\label{methods:repr}\index{\_\_repr\_\_()}

\begin{fulllineitems}
\phantomsection\label{methods:__repr__}\pysiglinewithargsret{\bfcode{\_\_repr\_\_}}{\emph{self}}{}
\end{fulllineitems}


Quaternions are represented in the algebraic way: \emph{q = a+bi+cj+dk}, and \emph{a, b, c, d} are floats. Components are of float type if \emph{self} is a quaternion, of integer type if \emph{self} is a Hurwitz quaternion.


\subsection{\_\_getitem\_\_()}
\label{methods:getitem}\index{\_\_getitem\_\_()}

\begin{fulllineitems}
\phantomsection\label{methods:__getitem__}\pysiglinewithargsret{\bfcode{\_\_getitem\_\_}}{\emph{self}, \emph{key}}{}
\end{fulllineitems}


Returns one of the four components of the quaternion. This method allows to get the components by

\code{quaternion{[}key{]}}


\subsection{\_\_setitem\_\_()}
\label{methods:setitem}\index{\_\_setitem\_\_()}

\begin{fulllineitems}
\phantomsection\label{methods:__setitem__}\pysiglinewithargsret{\bfcode{\_\_setitem\_\_}}{\emph{self}, \emph{key}, \emph{number}}{}
\end{fulllineitems}


Set one of the four components of the quaternion. This method allows to set the components by

\code{quaternion{[}key{]} = number}


\subsection{\_\_delitem\_\_()}
\label{methods:delitem}\index{\_\_delitem\_\_()}

\begin{fulllineitems}
\phantomsection\label{methods:__delitem__}\pysiglinewithargsret{\bfcode{\_\_delitem\_\_}}{\emph{self}, \emph{key}}{}
\end{fulllineitems}


Delete (set to zero) one of the four components of the quaternion. This method allows to write

\code{del quaternion{[}key{]}}


\subsection{\_\_delslice\_\_()}
\label{methods:delslice}\index{\_\_delslice\_\_()}

\begin{fulllineitems}
\phantomsection\label{methods:__delslice__}\pysiglinewithargsret{\bfcode{\_\_delslice\_\_}}{\emph{self}, \emph{key\_1}, \emph{key\_2}}{}
\end{fulllineitems}


Delete (set to zero) the components of the quaternion from the \emph{key\_1-th} to the \emph{key\_2-th}. This method allows to write

\code{del quaternion{[}1:3{]}}


\subsection{\_\_contains\_\_()}
\label{methods:contains}\index{\_\_contains\_\_()}

\begin{fulllineitems}
\phantomsection\label{methods:__contains__}\pysiglinewithargsret{\bfcode{\_\_contains\_\_}}{\emph{self}, \emph{key}}{}
\end{fulllineitems}


Returns 0 if the component of the quaternion with respect to \emph{key} is zero, 1 otherwise. This method allows to write

\code{del 'k' in quaternion}

\begin{Verbatim}[commandchars=\\\{\}]
\PYG{g+gp}{\textgreater{}\textgreater{}\textgreater{} }\PYG{n}{q} \PYG{o}{=} \PYG{n}{qmath}\PYG{o}{.}\PYG{n}{quaternion}\PYG{p}{(}\PYG{l+s}{'}\PYG{l+s}{1+1i+5k}\PYG{l+s}{'}\PYG{p}{)}
\PYG{g+gp}{\textgreater{}\textgreater{}\textgreater{} }\PYG{l+s}{'}\PYG{l+s}{j}\PYG{l+s}{'} \PYG{o+ow}{in} \PYG{n}{q}
\PYG{g+go}{False}
\PYG{g+gp}{\textgreater{}\textgreater{}\textgreater{} }\PYG{l+s}{'}\PYG{l+s}{i}\PYG{l+s}{'} \PYG{o+ow}{in} \PYG{n}{q}
\PYG{g+go}{True}
\end{Verbatim}


\section{Algebraic Operations}
\label{methods:algebraic-operations}

\subsection{\_\_eq\_\_()}
\label{methods:eq}\index{\_\_eq\_\_()}

\begin{fulllineitems}
\phantomsection\label{methods:__eq__}\pysiglinewithargsret{\bfcode{\_\_eq\_\_}}{\emph{self}, \emph{other}}{}
\end{fulllineitems}


Returns \code{True} if two quaternion are equal, \code{False} otherwise. This method allows to write

\code{quaternion1 == quaternion2}

\begin{Verbatim}[commandchars=\\\{\}]
\PYG{g+gp}{\textgreater{}\textgreater{}\textgreater{} }\PYG{n}{q} \PYG{o}{=} \PYG{n}{qmath}\PYG{o}{.}\PYG{n}{quaternion}\PYG{p}{(}\PYG{l+s}{'}\PYG{l+s}{1+1k}\PYG{l+s}{'}\PYG{p}{)}
\PYG{g+gp}{\textgreater{}\textgreater{}\textgreater{} }\PYG{n}{q} \PYG{o}{==} \PYG{l+m+mi}{0}
\PYG{g+go}{False}
\PYG{g+gp}{\textgreater{}\textgreater{}\textgreater{} }\PYG{n}{q} \PYG{o}{==} \PYG{l+s}{'}\PYG{l+s}{1+1k}\PYG{l+s}{'}
\PYG{g+go}{True}
\end{Verbatim}

Also equalities with a tolerance are admitted:

\begin{Verbatim}[commandchars=\\\{\}]
\PYG{g+gp}{\textgreater{}\textgreater{}\textgreater{} }\PYG{n}{q} \PYG{o}{==} \PYG{n}{qmath}\PYG{o}{.}\PYG{n}{quaternion}\PYG{p}{(}\PYG{p}{[}\PYG{l+m+mi}{1}\PYG{p}{,}\PYG{l+m+mi}{0}\PYG{p}{,}\PYG{l+m+mf}{1e-15}\PYG{p}{,}\PYG{l+m+mi}{1}\PYG{p}{]}\PYG{p}{)}\PYG{o}{\textbar{}}\PYG{l+m+mf}{1e-9}
\PYG{g+go}{True}
\PYG{g+gp}{\textgreater{}\textgreater{}\textgreater{} }\PYG{n}{q} \PYG{o}{==} \PYG{p}{[}\PYG{l+m+mi}{1}\PYG{p}{,}\PYG{l+m+mi}{1}\PYG{p}{,}\PYG{l+m+mf}{1e-15}\PYG{p}{,}\PYG{l+m+mi}{0}\PYG{p}{]}
\PYG{g+go}{False}
\end{Verbatim}


\subsection{\_\_ne\_\_()}
\label{methods:ne}\index{\_\_ne\_\_()}

\begin{fulllineitems}
\phantomsection\label{methods:__ne__}\pysiglinewithargsret{\bfcode{\_\_ne\_\_}}{\emph{self}, \emph{other}}{}
\end{fulllineitems}


Returns \code{False} if two quaternion are equal, \code{True} otherwise. This method allows to write

\code{quaternion1 != quaternion2}

\begin{Verbatim}[commandchars=\\\{\}]
\PYG{g+gp}{\textgreater{}\textgreater{}\textgreater{} }\PYG{n}{q} \PYG{o}{=} \PYG{n}{qmath}\PYG{o}{.}\PYG{n}{quaternion}\PYG{p}{(}\PYG{l+s}{'}\PYG{l+s}{1+1k}\PYG{l+s}{'}\PYG{p}{)}
\PYG{g+gp}{\textgreater{}\textgreater{}\textgreater{} }\PYG{n}{q} \PYG{o}{!=} \PYG{l+m+mi}{0}
\PYG{g+go}{True}
\PYG{g+gp}{\textgreater{}\textgreater{}\textgreater{} }\PYG{n}{q} \PYG{o}{!=} \PYG{l+s}{'}\PYG{l+s}{1+1k}\PYG{l+s}{'}
\PYG{g+go}{False}
\end{Verbatim}


\subsection{\_\_int\_\_()}
\label{methods:int}\index{\_\_int\_\_()}

\begin{fulllineitems}
\phantomsection\label{methods:__int__}\pysiglinewithargsret{\bfcode{\_\_int\_\_}}{\emph{self}}{}
\end{fulllineitems}


Converts \emph{self} components into integers.

\begin{Verbatim}[commandchars=\\\{\}]
\PYG{g+gp}{\textgreater{}\textgreater{}\textgreater{} }\PYG{n}{q} \PYG{o}{=} \PYG{n}{qmath}\PYG{o}{.}\PYG{n}{quaternion}\PYG{p}{(}\PYG{l+s}{'}\PYG{l+s}{1+1i+5k}\PYG{l+s}{'}\PYG{p}{)}
\PYG{g+gp}{\textgreater{}\textgreater{}\textgreater{} }\PYG{n}{q}
\PYG{g+go}{(1.0+1.0i+5.0k)}
\PYG{g+gp}{\textgreater{}\textgreater{}\textgreater{} }\PYG{n}{q}\PYG{o}{.}\PYG{n}{\PYGZus{}int\PYGZus{}}\PYG{p}{(}\PYG{p}{)}
\PYG{g+go}{(1+1i+5k)}
\end{Verbatim}


\subsection{\_\_iadd\_\_(), \_\_isub\_\_(), \_\_imul\_\_(), \_\_idiv\_\_()}
\label{methods:iadd-isub-imul-idiv}\index{\_\_iadd\_\_()}

\begin{fulllineitems}
\phantomsection\label{methods:__iadd__}\pysiglinewithargsret{\bfcode{\_\_iadd\_\_}}{\emph{self}, \emph{other}}{}
\end{fulllineitems}

\index{\_\_isub\_\_()}

\begin{fulllineitems}
\phantomsection\label{methods:__isub__}\pysiglinewithargsret{\bfcode{\_\_isub\_\_}}{\emph{self}, \emph{other}}{}
\end{fulllineitems}

\index{\_\_imul\_\_()}

\begin{fulllineitems}
\phantomsection\label{methods:__imul__}\pysiglinewithargsret{\bfcode{\_\_imul\_\_}}{\emph{self}, \emph{other}}{}
\end{fulllineitems}

\index{\_\_idiv\_\_()}

\begin{fulllineitems}
\phantomsection\label{methods:__idiv__}\pysiglinewithargsret{\bfcode{\_\_idiv\_\_}}{\emph{self}, \emph{other}}{}
\end{fulllineitems}


These methods are called to implement the augmented arithmetic assignments. These methods do the operation in-place (modifying self). If \emph{other} is a Hurwitz quaternion but its inverse is not, the \code{\_\_idiv\_\_()} method raises an error.

Methods \code{\_\_imul\_\_()} and \code{\_\_idiv\_\_()} depend on the choice of \emph{matrix} in the initialization of the quaternions.

\begin{Verbatim}[commandchars=\\\{\}]
\PYG{g+gp}{\textgreater{}\textgreater{}\textgreater{} }\PYG{n}{a} \PYG{o}{=} \PYG{n}{qmathcore}\PYG{o}{.}\PYG{n}{quaternion}\PYG{p}{(}\PYG{p}{[}\PYG{l+m+mi}{0}\PYG{p}{,}\PYG{o}{-}\PYG{l+m+mi}{1}\PYG{p}{,}\PYG{l+m+mi}{0}\PYG{p}{,}\PYG{o}{-}\PYG{l+m+mi}{1}\PYG{p}{]}\PYG{p}{,} \PYG{n}{matrix} \PYG{o}{=} \PYG{n}{np}\PYG{o}{.}\PYG{n}{array}\PYG{p}{(}\PYG{p}{[}\PYG{p}{[}\PYG{o}{-}\PYG{l+m+mi}{1}\PYG{p}{,}\PYG{l+m+mi}{0}\PYG{p}{,}\PYG{l+m+mi}{0}\PYG{p}{]}\PYG{p}{,}\PYG{p}{[}\PYG{l+m+mi}{0}\PYG{p}{,}\PYG{l+m+mi}{1}\PYG{p}{,}\PYG{l+m+mi}{0}\PYG{p}{]}\PYG{p}{,}\PYG{p}{[}\PYG{l+m+mi}{0}\PYG{p}{,}\PYG{l+m+mi}{0}\PYG{p}{,}\PYG{o}{-}\PYG{l+m+mi}{1}\PYG{p}{]}\PYG{p}{]}\PYG{p}{)}\PYG{p}{)}
\PYG{g+gp}{\textgreater{}\textgreater{}\textgreater{} }\PYG{n}{b} \PYG{o}{=} \PYG{n}{qmathcore}\PYG{o}{.}\PYG{n}{quaternion}\PYG{p}{(}\PYG{p}{[}\PYG{l+m+mi}{0}\PYG{p}{,}\PYG{l+m+mi}{0}\PYG{p}{,}\PYG{l+m+mi}{2}\PYG{p}{,}\PYG{l+m+mi}{0}\PYG{p}{]}\PYG{p}{,} \PYG{n}{matrix} \PYG{o}{=} \PYG{n}{np}\PYG{o}{.}\PYG{n}{array}\PYG{p}{(}\PYG{p}{[}\PYG{p}{[}\PYG{o}{-}\PYG{l+m+mi}{1}\PYG{p}{,}\PYG{l+m+mi}{0}\PYG{p}{,}\PYG{l+m+mi}{0}\PYG{p}{]}\PYG{p}{,}\PYG{p}{[}\PYG{l+m+mi}{0}\PYG{p}{,}\PYG{l+m+mi}{1}\PYG{p}{,}\PYG{l+m+mi}{0}\PYG{p}{]}\PYG{p}{,}\PYG{p}{[}\PYG{l+m+mi}{0}\PYG{p}{,}\PYG{l+m+mi}{0}\PYG{p}{,}\PYG{o}{-}\PYG{l+m+mi}{1}\PYG{p}{]}\PYG{p}{]}\PYG{p}{)}\PYG{p}{)}
\PYG{g+gp}{\textgreater{}\textgreater{}\textgreater{} }\PYG{n}{a} \PYG{o}{*}\PYG{o}{=} \PYG{n}{b}
\PYG{g+gp}{\textgreater{}\textgreater{}\textgreater{} }\PYG{n}{a}
\PYG{g+go}{(-2.0i+2.0k)}
\end{Verbatim}


\subsection{\_\_imod\_\_()}
\label{methods:imod}\index{\_\_imod\_\_()}

\begin{fulllineitems}
\phantomsection\label{methods:__imod__}\pysiglinewithargsret{\bfcode{\_\_imod\_\_}}{\emph{self}, \emph{other}}{}
\end{fulllineitems}


This method is called to implement the modular reduction. It is performed only if \emph{self} is a Hurwitz quaternion and \emph{other} is an integer. Otherwise an error is raised.


\subsection{\_\_add\_\_(), \_\_sub\_\_(), \_\_mul\_\_(), \_\_div\_\_(), \_\_mod\_\_()}
\label{methods:add-sub-mul-div-mod}\index{\_\_add\_\_()}

\begin{fulllineitems}
\phantomsection\label{methods:__add__}\pysiglinewithargsret{\bfcode{\_\_add\_\_}}{\emph{self}, \emph{other}}{}
\end{fulllineitems}

\index{\_\_sub\_\_()}

\begin{fulllineitems}
\phantomsection\label{methods:__sub__}\pysiglinewithargsret{\bfcode{\_\_sub\_\_}}{\emph{self}, \emph{other}}{}
\end{fulllineitems}

\index{\_\_mul\_\_()}

\begin{fulllineitems}
\phantomsection\label{methods:__mul__}\pysiglinewithargsret{\bfcode{\_\_mul\_\_}}{\emph{self}, \emph{other}}{}
\end{fulllineitems}

\index{\_\_div\_\_()}

\begin{fulllineitems}
\phantomsection\label{methods:__div__}\pysiglinewithargsret{\bfcode{\_\_div\_\_}}{\emph{self}, \emph{other}}{}
\end{fulllineitems}

\index{\_\_mod\_\_()}

\begin{fulllineitems}
\phantomsection\label{methods:__mod__}\pysiglinewithargsret{\bfcode{\_\_mod\_\_}}{\emph{self}, \emph{other}}{}
\end{fulllineitems}


These methods are called to implement the binary arithmetic operations. For instance, to evaluate the expression \code{x + y}, where \code{x} is a quaternion, \code{x.\_\_add\_\_(y)} is called. \code{y} can either be a quaternion (or a Hurwitz quaternion) or something that can be converted to quaternion.

If \emph{other} is a Hurwitz quaternion but its inverse is not, the \code{\_\_div\_\_} method raises an error.

The \code{\_\_mod\_\_} method is performed only if \emph{self} is a Hurwitz quaternion and \emph{other} is integer (see \code{\_\_imod\_\_}).

Methods \code{\_\_mul\_\_()} and \code{\_\_div\_\_()} depend on the choice of \emph{matrix} in the initialization of the quaternions (see \code{\_\_imul\_\_()} and \code{\_\_idiv\_\_()}).


\subsection{\_\_rmul\_\_(), \_\_rdiv\_\_()}
\label{methods:rmul-rdiv}\index{\_\_rmul\_\_()}

\begin{fulllineitems}
\phantomsection\label{methods:__rmul__}\pysiglinewithargsret{\bfcode{\_\_rmul\_\_}}{\emph{self}, \emph{other}}{}
\end{fulllineitems}

\index{\_\_rdiv\_\_()}

\begin{fulllineitems}
\phantomsection\label{methods:__rdiv__}\pysiglinewithargsret{\bfcode{\_\_rdiv\_\_}}{\emph{self}, \emph{other}}{}
\end{fulllineitems}


These methods are called to implement the binary arithmetic operations with reflected operands. If \emph{other} is a Hurwitz quaternion but its inverse is not, the \code{\_\_rdiv\_\_} method raises an error.

Methods \code{\_\_rmul\_\_()} and \code{\_\_rdiv\_\_()} depend on the choice of \emph{matrix} in the initialization of the quaternions (see \code{\_\_imul\_\_()} and \code{\_\_idiv\_\_()}).


\subsection{\_\_neg\_\_()}
\label{methods:neg}\index{\_\_neg\_\_()}

\begin{fulllineitems}
\phantomsection\label{methods:__neg__}\pysiglinewithargsret{\bfcode{\_\_neg\_\_}}{\emph{self}}{}
\end{fulllineitems}


Return the opposite of a quaternion. You can write \code{- quaternion}.


\subsection{\_\_pow\_\_()}
\label{methods:pow}\index{\_\_pow\_\_()}

\begin{fulllineitems}
\phantomsection\label{methods:__pow__}\pysiglinewithargsret{\bfcode{\_\_pow\_\_}}{\emph{self}, \emph{exponent}\optional{, \emph{modulo}}}{}
\end{fulllineitems}


Implements the operator \code{**}. The power of a quaternion can be computed for integer power (both positive or negative) and also if the exponent is half or third a number: for example square or cube roots are evaluated (see also \code{sqrt} and \code{croot}).

This method depends on the choice of \emph{matrix} in the initialization of the quaternions (see \code{\_\_imul\_\_()}).

If \emph{self} is a Hurwitz quaternion, powers are computed only for natural exponents.
Modular reduction is performed for Hurwitz quaternions (if \emph{self} is a Hamilton quaternion, modulo is ignored).

\begin{Verbatim}[commandchars=\\\{\}]
\PYG{g+gp}{\textgreater{}\textgreater{}\textgreater{} }\PYG{n}{base} \PYG{o}{=} \PYG{n}{qmath}\PYG{o}{.}\PYG{n}{quaternion}\PYG{p}{(}\PYG{l+s}{'}\PYG{l+s}{1+1i+2j-2k}\PYG{l+s}{'}\PYG{p}{)}
\PYG{g+gp}{\textgreater{}\textgreater{}\textgreater{} }\PYG{n}{base} \PYG{o}{*}\PYG{o}{*} \PYG{l+m+mi}{3}
\PYG{g+go}{(-26.0-6.0i-12.0j+12.0k)}
\PYG{g+gp}{\textgreater{}\textgreater{}\textgreater{} }\PYG{n}{base} \PYG{o}{*}\PYG{o}{*} \PYG{p}{(}\PYG{o}{-}\PYG{l+m+mi}{2}\PYG{p}{)}
\PYG{g+go}{(-0.08-0.02i-0.04j+0.04k)}
\PYG{g+gp}{\textgreater{}\textgreater{}\textgreater{} }\PYG{n}{qmath}\PYG{o}{.}\PYG{n}{quaternion}\PYG{p}{(}\PYG{p}{[}\PYG{o}{-}\PYG{l+m+mi}{5}\PYG{p}{,}\PYG{l+m+mi}{1}\PYG{p}{,}\PYG{l+m+mi}{0}\PYG{p}{,}\PYG{l+m+mi}{1}\PYG{p}{]}\PYG{p}{)} \PYG{o}{*}\PYG{o}{*} \PYG{p}{(}\PYG{l+m+mf}{1.0}\PYG{o}{/}\PYG{l+m+mi}{3}\PYG{p}{)}
\PYG{g+go}{(1.0+1.0i+1.0k)}
\PYG{g+gp}{\textgreater{}\textgreater{}\textgreater{} }\PYG{n}{qmath}\PYG{o}{.}\PYG{n}{quaternion}\PYG{p}{(}\PYG{p}{[}\PYG{o}{-}\PYG{l+m+mi}{5}\PYG{p}{,}\PYG{l+m+mi}{1}\PYG{p}{,}\PYG{l+m+mi}{0}\PYG{p}{,}\PYG{l+m+mi}{1}\PYG{p}{]}\PYG{p}{)} \PYG{o}{*}\PYG{o}{*} \PYG{p}{(}\PYG{l+m+mf}{2.0}\PYG{o}{/}\PYG{l+m+mi}{3}\PYG{p}{)}
\PYG{g+go}{(-1.0+2.0i+2.0k)}
\PYG{g+gp}{\textgreater{}\textgreater{}\textgreater{} }\PYG{n}{qmath}\PYG{o}{.}\PYG{n}{quaternion}\PYG{p}{(}\PYG{l+s}{'}\PYG{l+s}{1.0+1.0i+1.0k}\PYG{l+s}{'}\PYG{p}{)} \PYG{o}{*}\PYG{o}{*} \PYG{l+m+mi}{2}
\PYG{g+go}{(-1.0+2.0i+2.0k)}
\PYG{g+gp}{\textgreater{}\textgreater{}\textgreater{} }\PYG{n}{qmathcore}\PYG{o}{.}\PYG{n}{hurwitz}\PYG{p}{(}\PYG{l+s}{'}\PYG{l+s}{1+1i+1k}\PYG{l+s}{'}\PYG{p}{)} \PYG{o}{*}\PYG{o}{*} \PYG{l+m+mi}{2}
\PYG{g+go}{(-1+2i+2k)}
\PYG{g+gp}{\textgreater{}\textgreater{}\textgreater{} }\PYG{n+nb}{pow}\PYG{p}{(}\PYG{n}{qmathcore}\PYG{o}{.}\PYG{n}{hurwitz}\PYG{p}{(}\PYG{l+s}{'}\PYG{l+s}{1+1i+1k}\PYG{l+s}{'}\PYG{p}{)}\PYG{p}{,}\PYG{l+m+mi}{2}\PYG{p}{,}\PYG{l+m+mi}{3}\PYG{p}{)}
\PYG{g+go}{(2+2i+2k)}
\end{Verbatim}


\subsection{\_\_abs\_\_()}
\label{methods:abs}\index{\_\_abs\_\_()}

\begin{fulllineitems}
\phantomsection\label{methods:__abs__}\pysiglinewithargsret{\bfcode{\_\_abs\_\_}}{\emph{self}}{}
\end{fulllineitems}


Returns the modulus of the quaternion.


\subsection{equal()}
\label{methods:equal}\index{equal()}

\begin{fulllineitems}
\pysiglinewithargsret{\bfcode{equal}}{\emph{self}, \emph{other}\optional{, \emph{tolerance}}}{}
\end{fulllineitems}


Returns quaternion equality with arbitrary tolerance. If no tolerance is admitted it is the same as \code{\_\_eq\_\_(self,other)}.

\begin{Verbatim}[commandchars=\\\{\}]
\PYG{g+gp}{\textgreater{}\textgreater{}\textgreater{} }\PYG{n}{a} \PYG{o}{=} \PYG{n}{qmath}\PYG{o}{.}\PYG{n}{quaternion}\PYG{p}{(}\PYG{p}{[}\PYG{l+m+mi}{1}\PYG{p}{,}\PYG{l+m+mi}{1}\PYG{p}{,}\PYG{l+m+mf}{1e-15}\PYG{p}{,}\PYG{l+m+mi}{0}\PYG{p}{]}\PYG{p}{)}
\PYG{g+gp}{\textgreater{}\textgreater{}\textgreater{} }\PYG{n}{b} \PYG{o}{=} \PYG{n}{qmath}\PYG{o}{.}\PYG{n}{quaternion}\PYG{p}{(}\PYG{l+m+mi}{1}\PYG{o}{+}\PYG{l+m+mi}{1}\PYG{n}{j}\PYG{p}{)}
\PYG{g+gp}{\textgreater{}\textgreater{}\textgreater{} }\PYG{n}{a}\PYG{o}{.}\PYG{n}{equal}\PYG{p}{(}\PYG{n}{b}\PYG{p}{,}\PYG{l+m+mf}{1e-9}\PYG{p}{)}
\PYG{g+go}{True}
\PYG{g+gp}{\textgreater{}\textgreater{}\textgreater{} }\PYG{n}{a}\PYG{o}{.}\PYG{n}{equal}\PYG{p}{(}\PYG{n}{b}\PYG{p}{)}
\PYG{g+go}{False}
\end{Verbatim}


\section{Other Algebraic Manipulations}
\label{methods:other-algebraic-manipulations}

\subsection{real()}
\label{methods:real}\index{real()}

\begin{fulllineitems}
\pysiglinewithargsret{\bfcode{real}}{\emph{self}}{}
\end{fulllineitems}


Returns the real part of the quaternion.


\subsection{imag()}
\label{methods:imag}\index{imag()}

\begin{fulllineitems}
\pysiglinewithargsret{\bfcode{imag}}{\emph{self}}{}
\end{fulllineitems}


Returns the imaginary part of the quaternion.


\subsection{trace()}
\label{methods:trace}\index{trace()}

\begin{fulllineitems}
\pysiglinewithargsret{\bfcode{trace}}{\emph{self}}{}
\end{fulllineitems}


Returns the trace of the quaternion (the double of its real part).


\subsection{conj()}
\label{methods:conj}\index{conj()}

\begin{fulllineitems}
\pysiglinewithargsret{\bfcode{conj}}{\emph{self}}{}
\end{fulllineitems}


Returns the conjugate of the quaternion.


\subsection{norm()}
\label{methods:norm}\index{norm()}

\begin{fulllineitems}
\pysiglinewithargsret{\bfcode{norm}}{\emph{self}}{}
\end{fulllineitems}


Returns the norm of the quaternion (the square of the modulus).


\subsection{delta()}
\label{methods:delta}\index{delta()}

\begin{fulllineitems}
\pysiglinewithargsret{\bfcode{delta}}{\emph{self}}{}
\end{fulllineitems}


Returns the \emph{delta} of the quaternion, that is the opposite of the norm of the imaginary part of the quaternion (see \emph{{[}2{]}} for details).


\subsection{inverse()}
\label{methods:inverse}\index{inverse()}

\begin{fulllineitems}
\pysiglinewithargsret{\bfcode{inverse}}{\emph{self}\optional{, \emph{modulo}}}{}
\end{fulllineitems}


Quaternionic inverse, if it exists. It is equivalent to \code{quaternion ** (-1)}.

Modular inversion can be performed for Hurwitz quaternions (if \emph{self} is a Hamilton quaternion, modulo is ignored).

\begin{Verbatim}[commandchars=\\\{\}]
\PYG{g+gp}{\textgreater{}\textgreater{}\textgreater{} }\PYG{n}{a} \PYG{o}{=} \PYG{n}{qmath}\PYG{o}{.}\PYG{n}{quaternion}\PYG{p}{(}\PYG{p}{[}\PYG{l+m+mi}{2}\PYG{p}{,}\PYG{o}{-}\PYG{l+m+mi}{2}\PYG{p}{,}\PYG{o}{-}\PYG{l+m+mi}{4}\PYG{p}{,}\PYG{o}{-}\PYG{l+m+mi}{1}\PYG{p}{]}\PYG{p}{)}
\PYG{g+gp}{\textgreater{}\textgreater{}\textgreater{} }\PYG{n}{a}\PYG{o}{.}\PYG{n}{inverse}\PYG{p}{(}\PYG{p}{)}
\PYG{g+go}{(0.08+0.08i+0.16j+0.04k)}
\PYG{g+gp}{\textgreater{}\textgreater{}\textgreater{} }\PYG{n}{b} \PYG{o}{=} \PYG{n}{qmath}\PYG{o}{.}\PYG{n}{hurwitz}\PYG{p}{(}\PYG{p}{[}\PYG{l+m+mi}{0}\PYG{p}{,}\PYG{o}{-}\PYG{l+m+mi}{2}\PYG{p}{,}\PYG{o}{-}\PYG{l+m+mi}{2}\PYG{p}{,}\PYG{l+m+mi}{0}\PYG{p}{]}\PYG{p}{)}
\PYG{g+gp}{\textgreater{}\textgreater{}\textgreater{} }\PYG{n}{b}\PYG{o}{.}\PYG{n}{inverse}\PYG{p}{(}\PYG{l+m+mi}{13}\PYG{p}{)}
\PYG{g+go}{(10i+10j)}
\end{Verbatim}


\subsection{unitary()}
\label{methods:unitary}\index{unitary()}

\begin{fulllineitems}
\pysiglinewithargsret{\bfcode{unitary}}{\emph{self}}{}
\end{fulllineitems}


Returns the normalized quaternion, if different from zero.


\subsection{sqrt()}
\label{methods:sqrt}\index{sqrt()}

\begin{fulllineitems}
\pysiglinewithargsret{\bfcode{sqrt}}{\emph{self}}{}
\end{fulllineitems}


Computes the square root of the quaternion. If the quaternion has only two roots, the one with positive trace is given: if this method returns \code{r}, also \code{-r} is a root (see \footnote{\begin{enumerate}
\setcounter{enumi}{12}
\item {} 
Abrate, \href{http://www.worldscinet.com/jaa/08/0803/S0219498809003308.html}{Quadratic Formulas for Generalized Quaternions}, The Journal of Algebra and its Applications, Vol. 8, Issue 3 (2009), pp. 289-306.

\end{enumerate}
} and \footnote{\begin{enumerate}
\setcounter{enumi}{12}
\item {} 
Abrate \emph{The Roots of a Generalized Quaternion}, Congressus Numeratium, Vol. 201 (2010), pp. 179-186, Kluwer Academic Publishers.

\end{enumerate}
} for details on roots extraction).


\subsection{croot()}
\label{methods:croot}\index{croot()}

\begin{fulllineitems}
\pysiglinewithargsret{\bfcode{croot}}{\emph{self}}{}
\end{fulllineitems}


Computes the cube root (unique) of a quaternion (see \footnotemark[1] and \footnotemark[2] for details on roots extraction).


\subsection{QuaternionToRotation()}
\label{methods:quaterniontorotation}\index{QuaternonToRotation()}

\begin{fulllineitems}
\phantomsection\label{methods:QuaternonToRotation}\pysiglinewithargsret{\bfcode{QuaternonToRotation}}{\emph{self}}{}
\end{fulllineitems}


Converts the quaternion, if unitary, into a rotation matrix.
\paragraph{References}


\chapter{Utility Functions}
\label{functions::doc}\label{functions:utility-functions}

\section{Basic Functions}
\label{functions:basic-functions}

\subsection{real()}
\label{functions:real}\index{real() (built-in function)}

\begin{fulllineitems}
\pysiglinewithargsret{\bfcode{real}}{\emph{object}}{}
\end{fulllineitems}


The same as \code{object.real()}.


\subsection{imag()}
\label{functions:imag}\index{imag() (built-in function)}

\begin{fulllineitems}
\pysiglinewithargsret{\bfcode{imag}}{\emph{object}}{}
\end{fulllineitems}


The same as \code{object.imag()}.


\subsection{trace()}
\label{functions:trace}\index{trace() (built-in function)}

\begin{fulllineitems}
\pysiglinewithargsret{\bfcode{trace}}{\emph{object}}{}
\end{fulllineitems}


The same as \code{object.trace()}.


\subsection{conj()}
\label{functions:conj}\index{conj() (built-in function)}

\begin{fulllineitems}
\pysiglinewithargsret{\bfcode{conj}}{\emph{object}}{}
\end{fulllineitems}


The same as \code{object.conj()}.


\subsection{norm()}
\label{functions:norm}\index{norm() (built-in function)}

\begin{fulllineitems}
\pysiglinewithargsret{\bfcode{norm}}{\emph{object}}{}
\end{fulllineitems}


The same as \code{object.norm()}.


\subsection{delta()}
\label{functions:delta}\index{delta() (built-in function)}

\begin{fulllineitems}
\pysiglinewithargsret{\bfcode{delta}}{\emph{object}}{}
\end{fulllineitems}


The same as \code{object.delta()}.


\subsection{inverse()}
\label{functions:inverse}\index{inverse() (built-in function)}

\begin{fulllineitems}
\pysiglinewithargsret{\bfcode{inverse}}{\emph{object}\optional{, \emph{module}}}{}
\end{fulllineitems}


The same as \code{object.inverse({[}module{]})}.


\subsection{unitary()}
\label{functions:unitary}\index{unitary() (built-in function)}

\begin{fulllineitems}
\pysiglinewithargsret{\bfcode{unitary}}{\emph{object}}{}
\end{fulllineitems}


The same as \code{object.unitary()}.


\subsection{sqrt()}
\label{functions:sqrt}\index{sqrt() (built-in function)}

\begin{fulllineitems}
\pysiglinewithargsret{\bfcode{sqrt}}{\emph{object}}{}
\end{fulllineitems}


The same as \code{object.sqrt()} (see \footnote{\begin{enumerate}
\setcounter{enumi}{12}
\item {} 
Abrate, \href{http://www.worldscinet.com/jaa/08/0803/S0219498809003308.html}{Quadratic Formulas for Generalized Quaternions}, The Journal of Algebra and its Applications, Vol. 8, Issue 3 (2009), pp. 289-306.

\end{enumerate}
} and \footnote{\begin{enumerate}
\setcounter{enumi}{12}
\item {} 
Abrate \emph{The Roots of a Generalized Quaternion}, Congressus Numeratium, Vol. 201 (2010), pp. 179-186, Kluwer Academic Publishers.

\end{enumerate}
} for details on roots extraction).


\subsection{croot()}
\label{functions:croot}\index{croot() (built-in function)}

\begin{fulllineitems}
\pysiglinewithargsret{\bfcode{croot}}{\emph{object}}{}
\end{fulllineitems}


The same as \code{object.croot()} (see \footnotemark[1] and \footnotemark[2] for details on roots extraction).


\section{Additional Functions}
\label{functions:additional-functions}

\subsection{QuaternionToRotation()}
\label{functions:quaterniontorotation}\index{QuaternionToRotation() (built-in function)}

\begin{fulllineitems}
\phantomsection\label{functions:QuaternionToRotation}\pysiglinewithargsret{\bfcode{QuaternionToRotation}}{\emph{object}}{}
\end{fulllineitems}


The same as \code{object.QuaternioinToRotation()}.


\subsection{RotationToQuaternion()}
\label{functions:rotationtoquaternion}\index{RotationToQuaternion() (built-in function)}

\begin{fulllineitems}
\phantomsection\label{functions:RotationToQuaternion}\pysiglinewithargsret{\bfcode{RotationToQuaternion}}{\emph{angle}, \emph{vector}}{}
\end{fulllineitems}


Converts a pair angle-vector into a quaternion.


\subsection{StringToQuaternion()}
\label{functions:stringtoquaternion}\index{StringToQuaternion() (built-in function)}

\begin{fulllineitems}
\phantomsection\label{functions:StringToQuaternion}\pysiglinewithargsret{\bfcode{StringToQuaternion}}{\emph{string}}{}
\end{fulllineitems}


Converts a string into a quaternion.


\subsection{MatrixToEuler()}
\label{functions:matrixtoeuler}\index{MatrixToEuler() (built-in function)}

\begin{fulllineitems}
\phantomsection\label{functions:MatrixToEuler}\pysiglinewithargsret{\bfcode{MatrixToEuler}}{\emph{matrix}}{}
\end{fulllineitems}


Converts a 3X3 (numpy) matrix into a vector having Euler angles as components.


\subsection{EulerToQuaternion()}
\label{functions:eulertoquaternion}\index{EulerToQuaternion() (built-in function)}

\begin{fulllineitems}
\phantomsection\label{functions:EulerToQuaternion}\pysiglinewithargsret{\bfcode{EulerToQuaternion}}{\emph{list}}{}
\end{fulllineitems}


Converts a vector whose components are Euler angles into a quaternion.


\subsection{identity()}
\label{functions:identity}\index{identity() (built-in function)}

\begin{fulllineitems}
\pysiglinewithargsret{\bfcode{identity}}{}{}
\end{fulllineitems}


Returns 1 as a quaternion.

\begin{Verbatim}[commandchars=\\\{\}]
\PYG{g+gp}{\textgreater{}\textgreater{}\textgreater{} }\PYG{n}{qmath}\PYG{o}{.}\PYG{n}{identity}\PYG{p}{(}\PYG{p}{)}
\PYG{g+go}{(1.0)}
\end{Verbatim}


\subsection{zero()}
\label{functions:zero}\index{zero() (built-in function)}

\begin{fulllineitems}
\pysiglinewithargsret{\bfcode{zero}}{\emph{object}}{}
\end{fulllineitems}

\begin{description}
\item[{Returns 0 as a quaternion.}] \leavevmode
\begin{Verbatim}[commandchars=\\\{\}]
\PYG{g+gp}{\textgreater{}\textgreater{}\textgreater{} }\PYG{n}{qmath}\PYG{o}{.}\PYG{n}{zero}\PYG{p}{(}\PYG{p}{)}
\PYG{g+go}{(0.0)}
\end{Verbatim}

\end{description}


\subsection{dot()}
\label{functions:dot}\index{dot() (built-in function)}

\begin{fulllineitems}
\pysiglinewithargsret{\bfcode{dot}}{\emph{object\_1}, \emph{object\_2}}{}
\end{fulllineitems}


Returns the dot product of two quaternions.

\begin{Verbatim}[commandchars=\\\{\}]
\PYG{g+gp}{\textgreater{}\textgreater{}\textgreater{} }\PYG{n}{a} \PYG{o}{=} \PYG{n}{qmath}\PYG{o}{.}\PYG{n}{quaternion}\PYG{p}{(}\PYG{l+s}{'}\PYG{l+s}{1+2i-2k}\PYG{l+s}{'}\PYG{p}{)}
\PYG{g+gp}{\textgreater{}\textgreater{}\textgreater{} }\PYG{n}{b} \PYG{o}{=} \PYG{n}{qmath}\PYG{o}{.}\PYG{n}{quaternion}\PYG{p}{(}\PYG{l+s}{'}\PYG{l+s}{3-2i+8j}\PYG{l+s}{'}\PYG{p}{)}
\PYG{g+gp}{\textgreater{}\textgreater{}\textgreater{} }\PYG{n}{qmath}\PYG{o}{.}\PYG{n}{dot}\PYG{p}{(}\PYG{n}{a}\PYG{p}{,}\PYG{n}{b}\PYG{p}{)}
\PYG{g+go}{-1.0}
\end{Verbatim}


\subsection{CrossRatio()}
\label{functions:crossratio}\index{CrossRatio() (built-in function)}

\begin{fulllineitems}
\phantomsection\label{functions:CrossRatio}\pysiglinewithargsret{\bfcode{CrossRatio}}{\emph{a}, \emph{b}, \emph{c}, \emph{d}}{}
\end{fulllineitems}


Returns the cross ratio of four quaternions.

If \code{a} and \code{d} or \code{b} and \code{c} are equal, returns the string \code{'Infinity'}.

The arguments of \code{CrossRatio} can be passed as a tuple.

\begin{Verbatim}[commandchars=\\\{\}]
\PYG{g+gp}{\textgreater{}\textgreater{}\textgreater{} }\PYG{n}{a} \PYG{o}{=} \PYG{n}{qmath}\PYG{o}{.}\PYG{n}{quaternion}\PYG{p}{(}\PYG{p}{[}\PYG{l+m+mi}{1}\PYG{p}{,}\PYG{l+m+mi}{0}\PYG{p}{,}\PYG{l+m+mi}{1}\PYG{p}{,}\PYG{l+m+mi}{0}\PYG{p}{]}\PYG{p}{)}
\PYG{g+gp}{\textgreater{}\textgreater{}\textgreater{} }\PYG{n}{b} \PYG{o}{=} \PYG{n}{qmath}\PYG{o}{.}\PYG{n}{quaternion}\PYG{p}{(}\PYG{p}{[}\PYG{l+m+mi}{0}\PYG{p}{,}\PYG{l+m+mi}{1}\PYG{p}{,}\PYG{l+m+mi}{0}\PYG{p}{,}\PYG{l+m+mi}{1}\PYG{p}{]}\PYG{p}{)}
\PYG{g+gp}{\textgreater{}\textgreater{}\textgreater{} }\PYG{n}{c} \PYG{o}{=} \PYG{n}{qmath}\PYG{o}{.}\PYG{n}{quaternion}\PYG{p}{(}\PYG{p}{[}\PYG{o}{-}\PYG{l+m+mi}{1}\PYG{p}{,}\PYG{l+m+mi}{0}\PYG{p}{,}\PYG{o}{-}\PYG{l+m+mi}{1}\PYG{p}{,}\PYG{l+m+mi}{0}\PYG{p}{]}\PYG{p}{)}
\PYG{g+gp}{\textgreater{}\textgreater{}\textgreater{} }\PYG{n}{d} \PYG{o}{=} \PYG{n}{qmath}\PYG{o}{.}\PYG{n}{quaternion}\PYG{p}{(}\PYG{p}{[}\PYG{l+m+mi}{0}\PYG{p}{,}\PYG{o}{-}\PYG{l+m+mi}{1}\PYG{p}{,}\PYG{l+m+mi}{0}\PYG{p}{,}\PYG{o}{-}\PYG{l+m+mi}{1}\PYG{p}{]}\PYG{p}{)}
\PYG{g+gp}{\textgreater{}\textgreater{}\textgreater{} }\PYG{n}{qmath}\PYG{o}{.}\PYG{n}{CrossRatio}\PYG{p}{(}\PYG{n}{a}\PYG{p}{,}\PYG{n}{b}\PYG{p}{,}\PYG{n}{c}\PYG{p}{,}\PYG{n}{d}\PYG{p}{)}
\PYG{g+go}{(2.0)}
\PYG{g+gp}{\textgreater{}\textgreater{}\textgreater{} }\PYG{n}{tpl} \PYG{o}{=} \PYG{n}{a}\PYG{p}{,}\PYG{n}{b}\PYG{p}{,}\PYG{n}{c}\PYG{p}{,}\PYG{n}{d}
\PYG{g+gp}{\textgreater{}\textgreater{}\textgreater{} }\PYG{n}{qmath}\PYG{o}{.}\PYG{n}{CrossRatio}\PYG{p}{(}\PYG{n}{tpl}\PYG{p}{)}
\PYG{g+go}{(2.0)}
\PYG{g+gp}{\textgreater{}\textgreater{}\textgreater{} }\PYG{n}{qmath}\PYG{o}{.}\PYG{n}{CrossRatio}\PYG{p}{(}\PYG{n}{a}\PYG{p}{,}\PYG{n}{b}\PYG{p}{,}\PYG{n}{b}\PYG{p}{,}\PYG{n}{d}\PYG{p}{)}
\PYG{g+go}{'Infinity'}
\PYG{g+gp}{\textgreater{}\textgreater{}\textgreater{} }\PYG{n}{qmath}\PYG{o}{.}\PYG{n}{CrossRatio}\PYG{p}{(}\PYG{n}{a}\PYG{p}{,}\PYG{n}{a}\PYG{p}{,}\PYG{n}{a}\PYG{p}{,}\PYG{n}{d}\PYG{p}{)}
\PYG{g+go}{(1.0)}
\PYG{g+gp}{\textgreater{}\textgreater{}\textgreater{} }\PYG{n}{qmath}\PYG{o}{.}\PYG{n}{CrossRatio}\PYG{p}{(}\PYG{n}{a}\PYG{p}{,}\PYG{n}{b}\PYG{p}{,}\PYG{n}{a}\PYG{p}{,}\PYG{n}{b}\PYG{p}{)}
\PYG{g+go}{(0.0)}
\end{Verbatim}


\subsection{Moebius()}
\label{functions:moebius}\index{Moebius() (built-in function)}

\begin{fulllineitems}
\phantomsection\label{functions:Moebius}\pysiglinewithargsret{\bfcode{Moebius}}{\emph{z}, \emph{a}, \emph{b}, \emph{c}, \emph{d}}{}
\end{fulllineitems}


Returns the Moebius transformation with parameters a,b,c and d.

If \code{c * z + d == 0} returns the string \code{'Infinity'}.

The arguments of \code{Moebius} can be passed as a tuple.

\begin{Verbatim}[commandchars=\\\{\}]
\PYG{g+gp}{\textgreater{}\textgreater{}\textgreater{} }\PYG{n}{a} \PYG{o}{=} \PYG{n}{qmath}\PYG{o}{.}\PYG{n}{quaternion}\PYG{p}{(}\PYG{p}{[}\PYG{l+m+mi}{1}\PYG{p}{,}\PYG{l+m+mi}{1}\PYG{p}{,}\PYG{l+m+mi}{1}\PYG{p}{,}\PYG{l+m+mi}{0}\PYG{p}{]}\PYG{p}{)}
\PYG{g+gp}{\textgreater{}\textgreater{}\textgreater{} }\PYG{n}{b} \PYG{o}{=} \PYG{n}{qmath}\PYG{o}{.}\PYG{n}{quaternion}\PYG{p}{(}\PYG{p}{[}\PYG{o}{-}\PYG{l+m+mi}{2}\PYG{p}{,}\PYG{l+m+mi}{1}\PYG{p}{,}\PYG{l+m+mi}{0}\PYG{p}{,}\PYG{l+m+mi}{1}\PYG{p}{]}\PYG{p}{)}
\PYG{g+gp}{\textgreater{}\textgreater{}\textgreater{} }\PYG{n}{c} \PYG{o}{=} \PYG{n}{qmath}\PYG{o}{.}\PYG{n}{quaternion}\PYG{p}{(}\PYG{p}{[}\PYG{l+m+mi}{1}\PYG{p}{,}\PYG{l+m+mi}{0}\PYG{p}{,}\PYG{l+m+mi}{0}\PYG{p}{,}\PYG{l+m+mi}{0}\PYG{p}{]}\PYG{p}{)}
\PYG{g+gp}{\textgreater{}\textgreater{}\textgreater{} }\PYG{n}{d} \PYG{o}{=} \PYG{n}{qmath}\PYG{o}{.}\PYG{n}{quaternion}\PYG{p}{(}\PYG{p}{[}\PYG{l+m+mi}{0}\PYG{p}{,}\PYG{o}{-}\PYG{l+m+mi}{1}\PYG{p}{,}\PYG{o}{-}\PYG{l+m+mi}{3}\PYG{p}{,}\PYG{o}{-}\PYG{l+m+mi}{4}\PYG{p}{]}\PYG{p}{)}
\PYG{g+gp}{\textgreater{}\textgreater{}\textgreater{} }\PYG{n}{z} \PYG{o}{=} \PYG{n}{qmath}\PYG{o}{.}\PYG{n}{quaternion}\PYG{p}{(}\PYG{p}{[}\PYG{l+m+mi}{1}\PYG{p}{,}\PYG{l+m+mi}{1}\PYG{p}{,}\PYG{l+m+mi}{3}\PYG{p}{,}\PYG{l+m+mi}{4}\PYG{p}{]}\PYG{p}{)}
\PYG{g+gp}{\textgreater{}\textgreater{}\textgreater{} }\PYG{n}{qmath}\PYG{o}{.}\PYG{n}{Moebius}\PYG{p}{(}\PYG{n}{z}\PYG{p}{,}\PYG{n}{a}\PYG{p}{,}\PYG{n}{b}\PYG{p}{,}\PYG{n}{c}\PYG{p}{,}\PYG{n}{d}\PYG{p}{)}
\PYG{g+go}{(-5.0+7.0i+7.0k)}
\PYG{g+gp}{\textgreater{}\textgreater{}\textgreater{} }\PYG{n}{d} \PYG{o}{=} \PYG{o}{-} \PYG{n}{z}
\PYG{g+gp}{\textgreater{}\textgreater{}\textgreater{} }\PYG{n}{z} \PYG{o}{=} \PYG{n}{qmath}\PYG{o}{.}\PYG{n}{Moebius}\PYG{p}{(}\PYG{n}{z}\PYG{p}{,}\PYG{n}{a}\PYG{p}{,}\PYG{n}{b}\PYG{p}{,}\PYG{n}{c}\PYG{p}{,}\PYG{n}{d}\PYG{p}{)}
\PYG{g+gp}{\textgreater{}\textgreater{}\textgreater{} }\PYG{n}{z}
\PYG{g+go}{'Infinity'}
\PYG{g+gp}{\textgreater{}\textgreater{}\textgreater{} }\PYG{n}{qmath}\PYG{o}{.}\PYG{n}{Moebius}\PYG{p}{(}\PYG{n}{z}\PYG{p}{,}\PYG{n}{a}\PYG{p}{,}\PYG{n}{b}\PYG{p}{,}\PYG{n}{c}\PYG{p}{,}\PYG{n}{d}\PYG{p}{)}
\PYG{g+go}{(1.0+1.0i+1.0j)}
\end{Verbatim}
\paragraph{References}


\chapter{References}
\label{references:references}\label{references::doc}
{[}1{]} M. Abrate, \href{http://www.worldscinet.com/jaa/08/0803/S0219498809003308.html}{Quadratic Formulas for Generalized Quaternions}, The Journal of Algebra and its Applications, Vol. 8, Issue 3, pp. 289-306, 2009.

{[}2{]} M. Abrate, \emph{The Roots of a Generalized Quaternion}, Congressus Numeratium, Vol. 201, Jan. 2010, 179-187 (Winnipeg, Canada)



\renewcommand{\indexname}{Index}
\printindex
\end{document}
